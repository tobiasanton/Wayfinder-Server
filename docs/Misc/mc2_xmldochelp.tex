
% Table description for XML nodes and attributes
% Use this when documenting XML.

\newenvironment{xmltable}
% begin xml table
{
\begin{flushleft}
\begin{tabular}{ | l | l | m{7cm} |}
\hline
\rowcolor{black}\color{white}Name & \color{white}Type & \color{white}Description \\
\hline
}
% end xml table
{
\end{tabular}
\end{flushleft}
}

% \xmldesc command is used within xmltable to add a new row with 
% xml description for a node or attribute
\newcommand{\xmldesc}[3]{ #1 & #2 & #3 \\ \hline }

% Description for status tables
% Use this when describing statuses
\newenvironment{statustable}
% begin status table
{
\begin{flushleft}
\begin{tabular}{ | c | m{12cm} | }
\hline
\rowcolor{black}\color{white}Code & \color{white}Description \\
\hline
}
% end status table
{
\end{tabular}
\end{flushleft}
}

% \statdesc command is used within a statustable to add a new row with
% status value and description.
\newcommand{\statusdesc}[2]{ #1 & #2 \\ \hline }

% Description for enum tables
% Use this when describing enums
\newenvironment{valuetable}
% begin status table
{
\begin{flushleft}
\begin{tabular}{ | c | m{8cm} | }
\hline
\rowcolor{black}\color{white}Value & \color{white}Description \\
\hline
}
% end status table
{
\end{tabular}
\end{flushleft}
}

% \valuedesc command is used within a valuetable to add a new row with
% enum value and description
\newcommand{\valuedesc}[2]{ #1 & #2 \\ \hline }
