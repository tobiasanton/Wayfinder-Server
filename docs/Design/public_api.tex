% 
% 
% Copyright (c) 1999 - 2010, Vodafone Group Services Ltd
% All rights reserved.
% 
% Redistribution and use in source and binary forms, with or without modification, are permitted provided that the following conditions are met:
% 
%     * Redistributions of source code must retain the above copyright notice, this list of conditions and the following disclaimer.
%     * Redistributions in binary form must reproduce the above copyright notice, this list of conditions and the following disclaimer in the documentation and/or other materials provided with the distribution.
%     * Neither the name of the Vodafone Group Services Ltd nor the names of its contributors may be used to endorse or promote products derived from this software without specific prior written permission.
% 
% THIS SOFTWARE IS PROVIDED BY THE COPYRIGHT HOLDERS AND CONTRIBUTORS "AS IS" AND ANY EXPRESS OR IMPLIED WARRANTIES, INCLUDING, BUT NOT LIMITED TO, THE IMPLIED WARRANTIES OF MERCHANTABILITY AND FITNESS FOR A PARTICULAR PURPOSE ARE DISCLAIMED. IN NO EVENT SHALL THE COPYRIGHT HOLDER OR CONTRIBUTORS BE LIABLE FOR ANY DIRECT, INDIRECT, INCIDENTAL, SPECIAL, EXEMPLARY, OR CONSEQUENTIAL DAMAGES (INCLUDING, BUT NOT LIMITED TO, PROCUREMENT OF SUBSTITUTE GOODS OR SERVICES; LOSS OF USE, DATA, OR PROFITS; OR BUSINESS INTERRUPTION) HOWEVER CAUSED AND ON ANY THEORY OF LIABILITY, WHETHER IN CONTRACT, STRICT LIABILITY, OR TORT (INCLUDING NEGLIGENCE OR OTHERWISE) ARISING IN ANY WAY OUT OF THE USE OF THIS SOFTWARE, EVEN IF ADVISED OF THE POSSIBILITY OF SUCH DAMAGE.
%
%  A description of the API that will be used by users
%  to communicate with MC2
%
%  Please note that this appendix is supposed to be a technical
%  specification.
%


\chapter{Public API --- XML} \label{publicxmlapi}


This is a description of the public XML interface to the \mc-system 
developed at Wayfinder Systems AB. The interface allows users to 
retrieve and modify user data stored in the \mc-system.
In addition to the formal definition of the Application Programming 
Interface (API) some examples of the usage are also included in 
this document. 
It is assumed that the reader has some knowledge of XML and HTTP. 
More information about XML can be found at the World Wide Web 
Consortium (W3C) web site at http://www.w3.org/XML/ and more about HTTP
at http://www.w3.org/Protocols/\#Specs.


\section{Sending Request to the Server}

The request should be sent as an XML-document in the body of an POST
request to the URI \verb$/xmlfile$. The Content-Type should be 
\texttt{text/xml}.


\subsection{Authenticate}

The login and password for the user of the request are sent in an 
\texttt{auth} element as \texttt{login} and \texttt{password} attribute.

\index{auth, element}
\index{login, attribute}
\index{password, attribute}
\begin{verbatim}
<!ELEMENT auth EMPTY >
<!ATTLIST auth login    CDATA #IMPLIED 
               password CDATA #IMPLIED >
\end{verbatim}


\subsection{If something is wrong}


Status-code and status-message can be returned in the reply if something 
is wrong with the request or the \mc-system had an internal error. The
Status-code contains a number and the status-message a string describing
the error.

\begin{description}
\item[0] The request succeeded.
\item[-1] The general error code. There was a problem with the request.
\item[-2] The request was malformed, see \emph{status\_message}
          for detailed error.
\item[-3] The request timed out. There was an internal timeout while 
          processing the request.
\item[-201] Access denied. The user does not have access to the requested 
  service or data. 
\item[-202] Unknown user. The user does not exist.
\item[-203] Invalid login. The login and password does not match.
\item[-206] Expired user. The user no longer has access to the service.
\end{description}



\section{Positions}

The supported coordinate systems and formats.
%  Included from externalapi.tex and public_xml.tex
The supported coordinate systems and formats.

The WGS84 format is: (N|S|E|W) D(D*)$^{\circ}$ MM' SS[.dddd]'', as used in the coordinate ``\emph{N 69$^{\circ}$ 03' 35.7840'', E 24$^{\circ}$ 09' 58.8238''}''.
That is, first one letter for point of compass, from (N|S|E|W).
\emph{N} and \emph{S} are used for latitudes. \emph{E} and \emph{W} are used for longitudes.
Then D(D*) is the number of degrees, with one or more digits.
\emph{D} represents a digit in degrees.
Then MM, the number of minutes, with exactly two digits. Pad with leading zeroes to make this two digits.
\emph{M} represents a digit in minutes.
Then SS, the number of seconds, with exactly two digits. Pad with leading zeroes to make this two digits.
\emph{S} represents a digit in seconds.
Then an optional 4 digit decimal number [.dddd], the number of milliseconds.
\emph{d} represents a digit in milliseconds.


An example of a \texttt{position\_item} in the \emph{WGS84} format.\\
~\\
\hspace*{6mm}\texttt{<position\_item position\_system="WGS84">}\\
\hspace*{12mm}\texttt{<lat>N 55$^{\circ}$ 43' 01.9561"</lat>}\\
\hspace*{12mm}\texttt{<lon>E 13$^{\circ}$ 11' 20.1477"</lon>}\\
\hspace*{6mm}\texttt{</position\_item>}\\

The WGS84Rad format is a radian angle using the WGS84 coordinate system.

An example of a \texttt{position\_item} in the \emph{WGS84Rad} format.
\begin{verbatim}
   <position_item position_system="WGS84Rad">
      <lat>0.9724487650</lat>
      <lon>0.2301902519</lon>
   </position_item>
\end{verbatim}

The WGS84Deg format is a degree angle using the WGS84 coordinate system.

An example of a \texttt{position\_item} in the \emph{WGS84Deg} format.
\begin{verbatim}
   <position_item position_system="WGS84Deg">
      <lat>55.71721002</lat>
      <lon>13.18892992</lon>
   </position_item>
\end{verbatim}

The MC2 format is a number, digits only.

An example of a \texttt{position\_item} in the \emph{MC2} format.
\begin{verbatim}
   <position_item position_system="MC2">
      <lat>664732208</lat>
      <lon>157350063</lon>
   </position_item>
\end{verbatim}


\subsection{Position item}

\index{position\_item, element}
\index{position\_system\_t, entity}
\index{lat, element}
\index{lon, element}
\begin{verbatim}
<!ENTITY % position_system_t "(WGS84|WGS84Rad|WGS84Deg|MC2)">
<!ELEMENT position_item ( lat, lon )>
<!ATTLIST position_item position_system %position_system_t; #REQUIRED >
<!ELEMENT lat ( #PCDATA )>
<!ELEMENT lon ( #PCDATA )>
\end{verbatim}


\section{User Favorites}

Request for add, delete and synchronize user's favorites. 
~\\
% Included from externalapi.tex and public_xml.tex
\index{favorite, element}
\begin{verbatim}
<!ELEMENT favorite ( position_item, fav_info* )>
<!ATTLIST favorite
                    id                CDATA  #REQUIRED
                    name              CDATA  #REQUIRED
                    short_name        CDATA  #REQUIRED
                    description       CDATA  #REQUIRED
                    category          CDATA  #REQUIRED
                    map_icon_name     CDATA  #REQUIRED >
<!ELEMENT fav_info EMPTY >
<!ATTLIST fav_info  type  %poi_info_t; #REQUIRED
                    key   CDATA        #REQUIRED
                    value CDATA        #REQUIRED >
\end{verbatim}
\label{favorite}
Favorite describes a special place that the user commonly wants to
go to.
\begin{description}
\item[\emph{favorite}] A favorite place for the user.
\item[\emph{id}] The id that is used to identify the favorite. See also Section \ref{favoriteid} \emph{favorite\_id}.
\item[\emph{name}] The name of the favorite to show in selection list.
\item[\emph{short\_name}] A short name of the favorite for quick selection.
  Might not be available on all interfaces.
\item[\emph{description}] A text describing the favorite and/or additional
  information.
\item[\emph{category}] Used to group favorites together. Currently not used.
\item[\emph{map\_icon\_name}] The symbol to use for the favorite when drawn on
  maps. Currently not used.
\item[\emph{fav\_info}] The list of information elements for the favorite.
\end{description}



\subsection{User Favorites Request}

Used to sync a client with the \mc-system.

\index{user\_favorites\_request, element}
\begin{verbatim}
<!ELEMENT user_favorites_request ( favorite_id_list?,
                                   delete_favorite_id_list?,
                                   add_favorite_list? )>
<!ATTLIST user_favorites_request 
                              transaction_id ID #REQUIRED
                              sync_favorites    %bool; "true"
                              position_system   %position_system_t; "MC2" >
<!ELEMENT favorite_id_list ( favorite_id* )>
<!ELEMENT favorite_id ( #PCDATA )>
<!ELEMENT delete_favorite_id_list ( favorite_id+ )>
<!ELEMENT add_favorite_list ( favorite+ )>
\end{verbatim}
\begin{description}
\item[\emph{favorite\_id\_list}] The id's of the favorites that the user
  has.
  \index{favorite\_id\_list, element}
\item[\emph{delete\_favorite\_id\_list}] The id's of the favorites that the
  user has deleted.
  \index{delete\_favorite\_id\_list, element}
\item[\emph{add\_favorite\_list}] The favorites that the user has added.
  \index{add\_favorite\_list, element}
\item[\emph{sync\_favorites}] If the reply should contain the add and 
  delete favorites that is needed to synchronize the user, uses
  \emph{favorite\_id\_list}. Otherwise the reply only contains the added
  and deleted favorites in the request.
\item[\emph{position\_system}] The position\_system to use in the favorites
  in the reply.
\item[\emph{favorite\_id}] \label{favoriteid} An id of a favorite. Value is from an \emph{id} in a \emph{favorite} described in Section 
  \ref{favorite}.
  \index{favorite\_id, element}
\end{description}


\subsection{User Favorites Reply}

This is a reply with a list of the favorites that the user has, the 
\texttt{favorite\_id\_list} in the request, but is no longer present on
the server, sent in the \texttt{delete\_favorite\_id\_list}, and
a list of favorites that is new to the user, sent in the 
\texttt{add\_favorite\_list}.

\index{user\_favorites\_reply, element}
\begin{verbatim}
<!ELEMENT user_favorites_reply ( delete_favorite_id_list?,
                                 add_favorite_list? )>
<!ATTLIST user_favorites_reply transaction_id ID #REQUIRED>
\end{verbatim}
\begin{description}
\item[\emph{delete\_favorite\_id\_list}] The id's of the favorites that the
  user should remove.
  \index{delete\_favorite\_id\_list, element}
\item[\emph{add\_favorite\_list}] The favorites that the user should
  add.
  \index{add\_favorite\_list, element}
\end{description}


\newpage
\section{Examples}

Some examples of the API described above.
To use the examples simply replace \texttt{mylogin} with your login and
\texttt{mypassword} with your password.

\subsection{Get all the favorites}

This example is a request for all the favorites stored in the \mc-system.

\begin{verbatim}
<request>
  <auth login="mylogin" password="mypassword"/>
  <user_favorites_request transaction_id="U1"/>
</request>
\end{verbatim}

\subsection{Add a favorite}

This example adds a favorite and returns all the favorites stored in the 
\mc-system. The ``id'' of the favorite to add should be ``0''.

\begin{verbatim}
<request>
  <auth login="mylogin" password="mypassword"/>
  <user_favorites_request transaction_id="U1">
     <add_favorite_list>
        <favorite id="0" 
                  name="The favorite"
                  short_name="Fav"
                  description="The favorite favorite"
                  category=""
                  map_icon_name="" >
           <position_item position_system="WGS84Rad">
              <lat>0.9717091809</lat>
              <lon>0.2193076014</lon>
           </position_item>
        </favorite>
     </add_favorite_list>
  </user_favorites_request>
</request>
\end{verbatim}

\subsection{Delete a favorite}

This example deletes a favorite, identified by the ``id'', from the
\mc-system and returns all the favorites stored in the 
\mc-system.

\begin{verbatim}
<request>
  <auth login="mylogin" password="mypassword"/>
  <user_favorites_request transaction_id="U1">
     <delete_favorite_id_list>
        <favorite_id>1205963691</favorite_id> 
     </delete_favorite_id_list>
  </user_favorites_request>
</request>
\end{verbatim}


\subsection{Synchronize favorites}

This example shows a request to update, synchronize, your own list of 
favorites with the \mc-system{s}. The reply contains the favorites in
the \mc-system that is not in the \texttt{favorite\_id\_list}.


\begin{verbatim}
<request>
  <auth login="mylogin" password="mypassword"/>
  <user_favorites_request transaction_id="U1">
     <favorite_id_list>
        <favorite_id>2104799423</favorite_id>
        <favorite_id>830684142</favorite_id>
        <favorite_id>1596509004</favorite_id>
        <favorite_id>610826900</favorite_id>
        <favorite_id>12</favorite_id>
     </favorite_id_list>
  </user_favorites_request>
</request>
\end{verbatim}


\clearemptydoublepage
% xxx add index entries into the examples as well
% and the public.dtd if they were not verbatim-input
\section{Document Type Definition}
\label{DTD}
\index{Public DTD, Document Type Definition}
The formal definition of the XML documents that are sent to and from
the \mc-system.
\verbatiminput{../../Server/bin/public.dtd}
