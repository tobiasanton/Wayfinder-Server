% Included from externalapi.tex and public_xml.tex
\index{favorite, element}
\begin{verbatim}
<!ELEMENT favorite ( position_item, fav_info* )>
<!ATTLIST favorite
                    id                CDATA  #REQUIRED
                    name              CDATA  #REQUIRED
                    short_name        CDATA  #REQUIRED
                    description       CDATA  #REQUIRED
                    category          CDATA  #REQUIRED
                    map_icon_name     CDATA  #REQUIRED >
<!ELEMENT fav_info EMPTY >
<!ATTLIST fav_info  type  %poi_info_t; #REQUIRED
                    key   CDATA        #REQUIRED
                    value CDATA        #REQUIRED >
\end{verbatim}
\label{favorite}
Favorite describes a special place that the user commonly wants to
go to.
\begin{description}
\item[\emph{favorite}] A favorite place for the user.
\item[\emph{id}] The id that is used to identify the favorite. See also Section \ref{favoriteid} \emph{favorite\_id}.
\item[\emph{name}] The name of the favorite to show in selection list.
\item[\emph{short\_name}] A short name of the favorite for quick selection.
  Might not be available on all interfaces.
\item[\emph{description}] A text describing the favorite and/or additional
  information.
\item[\emph{category}] Used to group favorites together. Currently not used.
\item[\emph{map\_icon\_name}] The symbol to use for the favorite when drawn on
  maps. Currently not used.
\item[\emph{fav\_info}] The list of information elements for the favorite.
\end{description}
